% Air quality has greatly improved over the last two decades due to stricter environmental
% policies and more efficient combustion engines. This change has been documented by the 
% Ozone Monitoring Instrument onboard the Aura satellite which has been operational since 2004. 

\textbf{Code on \href{https://github.com/Ovewh/ATM325/tree/master/Exercise6}{GitHub}}

\section*{Global change in \ch{NO2} and \ch{SO2} emission between 2005 and 2019} 
\Cref{fig:OMI_global_2005NO2} shows the time averaged global map of tropospheric 
\ch{NO2} for 2005. Regions of high \ch{NO2} in 2005 are mostly
anthropogenic, concentrated around the major cities. We see elevated values of tropospheric \ch{NO2} over 
China, Central Europe, Eastern United Sates, and South Africa. There 
are some natural sources of \ch{NO2} over the Amazon and the African rain forests which are due 
to wildfires.  

\Cref{fig:OMI_global_2019NO2} shows the 2019 time averaged global tropospheric
map  \ch{NO2}. Comparing 2019 with 2005 there is a major 
decrease in \ch{NO2} emission especially over Europe and the United Sates. China is also 
showing a noticeable decrease since 2005. However India which is currently in undergoing 
major industrialization and shows increased \ch{NO2} emission in 2019 compared to 2005. 

\begin{figure}[htbp]
    \centering
        \includegraphics[width=0.9\textwidth]{../figs/OMNO2d_2005.png}
    \caption{Time averaged map of \ch{NO2} tropospheric from 2005 NO2 from the OMI instrument
    onboard the Aura satellite   }
    \label{fig:OMI_global_2005NO2}
\end{figure}

\begin{figure}[htpb]
    \centering
        \includegraphics[width=0.9\textwidth]{../figs/OMNO2d_2019.png}
    \caption{Time averaged map of NO2 tropospheric from 2019 NO2 from the OMI instrument  
    onboard the Aura satellite}
    \label{fig:OMI_global_2019NO2}
\end{figure}

The \ch{SO2} have large natural sources due to volcanoes in addition to the
anthropogenic sources. The largest volcanic sources of \ch{SO2} 
in 2005 (\Cref{fig:OMI_global_2005SO2}) were the Anatahan (Northern Mariana 
islands in the Pacific Ocean), Nyiragongo (Central Congo) and Galapagos 
(in the Pacific Ocean 1200 km west of Ecuador). Considering the anthropogenic 
emission of \ch{SO2} the most noticeable source regions were China 
,the Nornickel smelters in Russia, industrial regions in South Africa and 
the Eastern United States.   

In 2019 \Cref{fig:OMI_global_2005SO2} there were sources of SO2 with volcanic origin in 
Alaska, in  the Andes mountain range, Nyiragongo and Papua New Guinea. The most interesting 
changes since 2005 is happening with anthropogenic emissions. China particularly show large 
reduction in \ch{SO2} emissions. The US has also a noticeable reduction in emissions. India 
however shows increased \ch{SO2}, due to the establishment of new coal fueled power plants. The 
emissions from the large point source in Russia (the Nornickel smelter) and South Africa 
seems more or less unchanged. 
\begin{figure}[htbp]
    \centering
        \includegraphics[width=0.9\textwidth]{../figs/OMSO2e_2005.png}
    \caption{Time averaged map global map of boundary layer \ch{SO2} for 2005, retrieved from OMI}
    \label{fig:OMI_global_2005SO2}
\end{figure}

\begin{figure}[htbp]
    \centering
        \includegraphics[width=0.9\textwidth]{../figs/OMSO2e_2019.png}
    \caption{Time averaged map global map of boundary layer \ch{SO2} for 2019, retrieved from OMI}
    \label{fig:OMI_global_2019SO2}
\end{figure}

\subsection*{Changes in \ch{NO2} and \ch{NO2} over China between 2005 and 2006}

Taking a closer look at the changes in tropospheric \ch{NO2} and \ch{SO2} over China
between 2005 and 2019 (\Cref{fig:OMI_China2005_SO2,fig:OMI_ChinaNO2,
fig:OMI_China2019_SO2,fig:OMI_China_2019NO2}). In 2005 (\Cref{fig:OMI_ChinaNO2})
there were bright hot spots of \ch{NO2} located
over the major cities in China with average tropospheric
\ch{NO2} above 2.5e16 \si{cm^{-1}}. The situation was not any better for \ch{SO2} (\Cref{fig:OMI_China2005_SO2}), with large \ch
{SO2} emission over the Eastern Central China due to the large number of coal power plants in 
the region. The bad air quality had severe consequences 
for peoples health and is estimated to kill about 1 million people in 
China every year \parencite{yue2020stronger}. As a measure to improve the 
air quality the Chinese government implemented the Air Pollution 
Prevention and Control Action Plan (APPCAP) from 2013 to 2017. Which 
aimed to reduce the PM 2.5 concentration by 10 \% - 25\% by 2017 in the major cities across China.      

Looking at the situation in 2019 (\Cref{fig:OMI_China2019_SO2,fig:OMI_China_2019NO2}) 
we see a noticeable reduction in both \ch{NO2} and \ch{SO2}, particularly the \ch{SO2} which 
in 2019 was less than 50 \% of the 2005 emissions. It does seem that measures of the APPCAP 
have worked especially for \ch{SO2}. \ch{NO2} did not show as a drastic reduction as the \ch
{SO2} and still have quite large values of time averaged yearly \ch{NO2} around the large cities. 
\begin{figure}[htpb]
    \centering
    \includegraphics[width=0.7\textwidth]{../figs/China_NO2_2005.png}
    \caption{Time averaged map of tropospheric \ch{NO2} from 2005 over China
    from the OMI instrument onboard the Aura satellite}
    \label{fig:OMI_ChinaNO2}
\end{figure}


\begin{figure}[htpb]
    \centering
    \includegraphics[width=0.7\textwidth]{../figs/China_SO2_2005.png}
    \caption{Time averaged map of boundary layer\ch{SO2} from 2005 over China
    from the OMI instrument onboard the Aura satellite}
    \label{fig:OMI_China2005_SO2}
\end{figure}

\begin{figure}[htpb]
    \centering
        \includegraphics[width=0.7\textwidth]{../figs/China_NO2_2019.png}
    \caption{Time averaged map of tropospheric \ch{NO2} from 2019 
    over China from the OMI instrument  
    onboard the Aura satellite}
    \label{fig:OMI_China_2019NO2}
\end{figure}


\begin{figure}[htpb]
    \centering
    \includegraphics[width=0.7\textwidth]{../figs/China_SO2_2019.png}
    \caption{Time averaged map of boundary layer\ch{SO2} from 2005 over China
    from the OMI instrument onboard the Aura satellite}
    \label{fig:OMI_China2019_SO2}
\end{figure}

\section*{OMI, TROPOMI comparison}
The TROPOspheric Monitoring Instrument (TROPOMI) onboard the Sentinel-5 Precursor 
satellite lunched in October 2017. TROPOMI extends the capabilities of OMI 
offering a small pixel size of 7km $\times$ 7km and swath of 2600km. The small 
pixel lets TROPOMI capture local structures which are too small for to be 
detected by OMI's larger pixel of 13 $\times$ 24 km. 
\Cref{fig:TROPOMI_jan_2019,fig:TROPOMI_apr_2019,fig:TROPOMI_july_2019,fig:TROPOMI_oct_2019} shows the map of 
monthly average tropospheric \ch{NO2} retrieved from TROPOMI for January, April, 
July and October 2019 respectively and \Cref{fig:OMI_jan_2019,fig:OMI_apr_2019,fig:OMI_july_2019,fig:OMI_oct_2019} 
shows the corresponding maps retrieved from OMI. 
The increased amount of details in the TROPOMI maps is considerable when 
comparing to OMI, in the TROPOMI maps you can clearly distinguish the cities from 
each other. This is particularly apparent if we look over China in  January, 
in OMI map everything appearers as a bright yellow blob, but with TROPOMI you can 
clearly see features within. 

Looking at the seasonality of tropospheric \ch{NO2}, we see that the anthropogenic 
\ch{NO2} are highest during winter (\Cref{fig:TROPOMI_jan_2019}). This is due to a larger 
demand for indoors heating, the boundary layer being more stable 
during winter (less mixing) and the chemical removal processes of \ch{NOX} being less efficient.
The anthropogenic 
\ch{NO2 }at least in the northern hemisphere is minimum during the summer (\cref
{fig:TROPOMI_july_2019}). 
The natural emissions of \ch{NO2} also have a maximum during winter, most likely due to 
there being more in bio mass burning happening during the winter season. This is case for the elevated \ch{NO2}
over the African continent between equator and the Sahara region.  However over Africa between equator 
and 20\degree south we see the opposite pattern with maximum in \ch{NO2} in July. 

\begin{figure}[htpb]
    \centering
    \includegraphics[width=0.8\textwidth]{../figs/TROPOMI_2019_January.png}
    \caption{Time averaged monthly map of TROPOMI tropospheric \ch{NO2} from January 2019}
    \label{fig:TROPOMI_jan_2019}
\end{figure}

\begin{figure}[htpb]
    \centering
    \includegraphics[width=0.8\textwidth]{../figs/TROPOMI_2019_April.png}
    \caption{Time averaged monthly map of TROPOMI tropospheric \ch{NO2} for April 2019}
    \label{fig:TROPOMI_apr_2019}
\end{figure}

\begin{figure}[htpb]
    \centering
    \includegraphics[width=0.8\textwidth]{../figs/TROPOMI_2019_July.png}
    \caption{Time averaged monthly map of TROPOMI tropospheric \ch{NO2} for July 2019}
    \label{fig:TROPOMI_july_2019}
\end{figure}

\begin{figure}[htpb]
    \centering
    \includegraphics[width=0.8\textwidth]{../figs/TROPOMI_2019_October.png}
    \caption{Time averaged monthly map of TROPOMI tropospheric \ch{NO2} for October 2019}
    \label{fig:TROPOMI_oct_2019}
\end{figure}

\begin{figure}
    \centering
    \includegraphics[width=0.8\textwidth]{../figs/OMI_2019_January.png}
    \caption{Time averaged monthly map of OMI tropospheric \ch{NO2} for January 2019}
    \label{fig:OMI_jan_2019}
\end{figure}

\begin{figure}
    \centering
    \includegraphics[width=0.8\textwidth]{../figs/OMI_2019_April.png}
    \caption{Time averaged monthly map of OMI tropospheric \ch{NO2} for April 2019}
    \label{fig:OMI_apr_2019}
\end{figure}

\begin{figure}
    \centering
    \includegraphics[width=0.8\textwidth]{../figs/OMI_2019_July.png}
    \caption{Time averaged monthly map of OMI tropospheric \ch{NO2} for July 2019}
    \label{fig:OMI_july_2019}
\end{figure}


\begin{figure}
    \centering
    \includegraphics[width=0.8\textwidth]{../figs/OMI_2019_October.png}
    \caption{Time averaged monthly map of OMI tropospheric \ch{NO2} for October 2019}
    \label{fig:OMI_oct_2019}
\end{figure}