Air quality has greatly improved over the last two decades due to stricter environmental
policies and more efficient combustion engines. This change has been documented by the 
Ozone Monitoring in  
\section*{1.} 
\Cref{fig:OMI_global_2005NO2} shows the time averaged global map of tropospheric 
\ch{NO2} for 2005. There are regions with high emissions particularly over 
China, Central Europe, Eastern United Sates, and in the industrial region of South Africa. 

\Cref{fig:OMI_global_2019NO2} shows the 2019 time averaged global tropospheric
map  \ch{NO2}. Comparing 2019 with 2005 there is a major 
decrease in \ch{NO2} emission especially over Europe and the United Sates. China is also 
showing a noticeable decrease since 2005. However India which is currently in undergoing 
major industrialization shows increased \ch{NO2} emission in 2019 compared to 2005. 

\begin{figure}[htbp]
    \centering
        \includegraphics[width=0.9\textwidth]{../figs/OMNO2d_2005.png}
    \caption{Time averaged map of \ch{NO2} tropospheric from 2005 NO2 from the OMI instrument
    onboard the Aura satellite   }
    \label{fig:OMI_global_2005NO2}
\end{figure}

\begin{figure}[htpb]
    \centering
        \includegraphics[width=0.9\textwidth]{../figs/OMNO2d_2019.png}
    \caption{Time averaged map of NO2 tropospheric from 2019 NO2 from the OMI instrument  
    onboard the Aura satellite}
    \label{fig:OMI_global_2019NO2}
\end{figure}
Taking a closer look at the changes in tropospheric \ch{NO2} over China
between 2005 and 2019 (\Cref{fig:OMI_global_2005NO2} and \Cref
{fig:OMI_China_2019NO2}). In 2005 there were bright hot spots located
over the major cities in China with average tropospheric
\ch{NO2} 2.5e16 \si{cm^{-1}}. The bad air quality has severe consequences 
for peoples health and is estimated to kill about 1 million people in 
China each year \parencite{yue2020stronger}. A measure to improved the 
air quality the Chinese government implemented the Air Pollution 
Prevention and Control Action Plan (APPCAP) from 2013 to 2017. Which 
aimed to reduce the PM 2.5 concentration by 10 \% - 25\% by 2017 in the major cities across China.      
\begin{figure}[htpb]
    \centering
    \includegraphics[width=0.7\textwidth]{../figs/China_NO2_2005.png}
    \caption{Time averaged map of tropospheric \ch{NO2} from 2005 over 
    from the OMI instrument onboard the Aura satellite}
    \label{fig:OMI_ChinaNO2}
\end{figure}

\begin{figure}[htpb]
    \centering
        \includegraphics[width=0.7\textwidth]{../figs/China_NO2_2019.png}
    \caption{Time averaged map of tropospheric \ch{NO2} from 2019 
    over China from the OMI instrument  
    onboard the Aura satellite}
    \label{fig:OMI_China_2019NO2}
\end{figure}

