Air quality has greatly improved over the last two decades due to stricter environmental
policies and more efficient combustion engines. This change has been documented by the 
Ozone Monitoring Instrument onboard the Aura satellite which has been operational since 2004.   
\section*{Global change in \ch{NO2} and \ch{SO2} emission between 2005 and 2019} 
\Cref{fig:OMI_global_2005NO2} shows the time averaged global map of tropospheric 
\ch{NO2} for 2005. High emission regions in 2005 are  
China, Central Europe, Eastern United Sates, and also industrial region of South Africa. 

\Cref{fig:OMI_global_2019NO2} shows the 2019 time averaged global tropospheric
map  \ch{NO2}. Comparing 2019 with 2005 there is a major 
decrease in \ch{NO2} emission especially over Europe and the United Sates. China is also 
showing a noticeable decrease since 2005. However India which is currently in undergoing 
major industrialization shows increased \ch{NO2} emission in 2019 compared to 2005. 

\begin{figure}[htbp]
    \centering
        \includegraphics[width=0.9\textwidth]{../figs/OMNO2d_2005.png}
    \caption{Time averaged map of \ch{NO2} tropospheric from 2005 NO2 from the OMI instrument
    onboard the Aura satellite   }
    \label{fig:OMI_global_2005NO2}
\end{figure}

\begin{figure}[htpb]
    \centering
        \includegraphics[width=0.9\textwidth]{../figs/OMNO2d_2019.png}
    \caption{Time averaged map of NO2 tropospheric from 2019 NO2 from the OMI instrument  
    onboard the Aura satellite}
    \label{fig:OMI_global_2019NO2}
\end{figure}

The \ch{SO2} have large natural sources due to volcanoes in addition to the
anthropogenic sources. The large volcanic sources of \ch{SO2} 
in 2005 \Cref{fig:OMI_global_2005SO2} were the Anatahan (Northern Mariana 
islands in the Pacific Ocean), Nyiragongo (Central Congo), Galapagos 
(in the Pacific Ocean 1200 km west of Ecuador) volcanos. The most 
noticeable anthropogenic sources regions of \ch{SO2} is China, the 
Nornickel smelters in Russia, industrial region in South Africa and the 
Eastern United States.   

In 2019 there were sources of SO2 with volcanic origin in Alaska, in the 
Andes, Nyiragongo and Papua New Guinea. For the anthropogenic sources 
there is are large reduction of China and Eastern United States. The \ch{SO2} emissions over South Africa and the Nornickel smelters seems to remain more or less the same. 
\begin{figure}[htbp]
    \centering
        \includegraphics[width=0.9\textwidth]{../figs/OMSO2e_2005.png}
    \caption{Time averaged map global map of boundary layer \ch{SO2} for 2005, retrieved from OMI}
    \label{fig:OMI_global_2005SO2}
\end{figure}

\begin{figure}[htbp]
    \centering
        \includegraphics[width=0.9\textwidth]{../figs/OMSO2e_2019.png}
    \caption{Time averaged map global map of boundary layer \ch{SO2} for 2005, retrieved from OMI}
    \label{fig:OMI_global_2019SO2}
\end{figure}


Taking a closer look at the changes in tropospheric \ch{NO2} over China
between 2005 and 2019 (\Cref{fig:OMI_global_2005NO2} and \Cref
{fig:OMI_China_2019NO2}). In 2005 there were bright hot spots located
over the major cities in China with average tropospheric
\ch{NO2} 2.5e16 \si{cm^{-1}}. The bad air quality has severe consequences 
for peoples health and is estimated to kill about 1 million people in 
China each year \parencite{yue2020stronger}. A measure to improved the 
air quality the Chinese government implemented the Air Pollution 
Prevention and Control Action Plan (APPCAP) from 2013 to 2017. Which 
aimed to reduce the PM 2.5 concentration by 10 \% - 25\% by 2017 in the major cities across China.      
\begin{figure}[htpb]
    \centering
    \includegraphics[width=0.7\textwidth]{../figs/China_NO2_2005.png}
    \caption{Time averaged map of tropospheric \ch{NO2} from 2005 over 
    from the OMI instrument onboard the Aura satellite}
    \label{fig:OMI_ChinaNO2}
\end{figure}

\begin{figure}[htpb]
    \centering
        \includegraphics[width=0.7\textwidth]{../figs/China_NO2_2019.png}
    \caption{Time averaged map of tropospheric \ch{NO2} from 2019 
    over China from the OMI instrument  
    onboard the Aura satellite}
    \label{fig:OMI_China_2019NO2}
\end{figure}

