Air quality has greatly improved over the last two decades due to stricter environmental
policies and more efficient combustion engines. This change has been documented by the 
Ozone Monitoring in  
\section*{1.} 
\Cref{fig:OMI_global_2005NO2} shows the time averaged global map of tropospheric 
\ch{NO2} emission during 2005. There are regions with high emissions particularly over 
China, Central Europe, Eastern United Sates, and in the industrial region of South Africa. 
Looking at the same map, but for 2019 (\Cref{fig:OMI_global_2019NO2}), there is a major 
decrease in \ch{NO2} emission especially over Europe and the United Sates. China is also 
showing a noticeable decrease since 2005. However India which is currently in undergoing 
major industrialization shows increased \ch{NO2} emission in 2019 compared to 2005.  
\begin{figure}
    \centering
        \includegraphics[width=0.9\textwidth]{../figs/OMNO2d_2005.png}
    \caption{Time averaged map of \ch{NO2} tropospheric from 2005 NO2 from the OMI instrument
    onboard the Aura satellite   }
    \label{fig:OMI_global_2005NO2}
\end{figure}

\begin{figure}
    \centering
        \includegraphics[width=0.9\textwidth]{../figs/OMNO2d_2019.png}
    \caption{Time averaged map of NO2 tropospheric from 2019 NO2 from the OMI instrument  
    onboard the Aura satellite}
    \label{fig:OMI_global_2019NO2}
\end{figure}

