\section*{1.}

\subsubsection*{a)} \Cref{fig:Time_series_XCH4_XCO2} show the time series of
daily average total column \ch{CO2} and \ch{CH4} from the Caltec TCCON station.
The \ch{XCO2} series has a 
clear seasonal cycle with a peak around February, before the photosynthesis
kicks in. The \ch{XCH4} also exhibit a
seasonal behaviour, with a peak in \ch{XCH4} around November, however the
\ch{XCH4} has more variation. The \ch{XCH4} in November is probably due to the
sink being weaker due to less sunlight, but the sources are still large, since
it is not yet cold enough for the for the organic material in the peat-lands to
be frozen. 

The average annual growth rate during this period for \ch{XCO2} and \ch{XCH4} is
about 2.2 ppm/year and 5778 ppb/year respectively.  
 
\begin{figure}[htbp]
    \includegraphics[width=\textwidth]{../Time_series.pdf}
    \centering
    \caption{Time series of \ch{XCO2} \textbf{(a)}, and \ch{XCH4} \textbf{(b)} from 2013 until 2020, from the Caltec TCCON station.}
    \label{fig:Time_series_XCH4_XCO2}

\end{figure}

\subsection*{b)}
\Cref{fig:scatter_XCO2} shows a scatter plot of \ch{XCO2}, where observations
from the GOSAT satellite is compared against ground based observations from
the Caltec TCCON station using two different geometric co-locations. The 
small region  corresponds to a tight co-location of $\pm$ 5 \degree in longitude
and $\pm$ 2.5 \degree in latitude from the TCCON site. The large region
corresponds to a wider co-location of $\pm$ 10 \degree in longitude and $\pm$
5 \degree in latitude. Comparing the large and small region with each other
there is not much difference.  In both cases GOSAT is slightly over estimating
the \ch{XCO2} relative to the TCCON observations. Still given the overestimation
the correlation is quite good in both for the small and large co-location with
correlation coefficions of 0.95, and 0.96 respectively. The small co-location
has a slightly smaller mean bias of 1.91 ppm compared to large co-location with
a mean bias of 2.05 ppm. The small difference between the small and large
co-location relates to how well mixed \ch{CO2} is in the atmosphere. Suggesting
that the observed bias in the GOSAT data is most likely because due to errors in
the retrievals and not local sources or sinks affecting the TCCON station,which
are not captured by in large co-location. Looking at the standard deviation the
for the GOSAT observations within the two regions, they are more or less equal,
further demonstrating how well mixed \ch{CO2} is.      
\begin{figure}[htbp]
    \centering
    \includegraphics[width=\textwidth]{../scatter_plot_XCO2.pdf}
    \caption{Scatter plot of daily mean \ch{XCO2} from GOSAT (x-axis) and TCCON caltec (y-axis). The red dotted is a least squares regression line of the data. \textbf{(a)}, corresponds to a tight geometric co-location, whereas \textbf{(b)} corresponds to a wider geometric co-location}
    \label{fig:scatter_XCO2}
\end{figure}

\subsection*{c)}

\textbf{Codes Here}\footnote{https://github.com/Ovewh/ATM325/tree/master/Exercise5}