\section*{1.}

\subsubsection*{a)} \Cref{fig:Time_series_XCH4_XCO2} show the time series of
daily average total column \ch{CO2} and \ch{CH4} from the Caltec TCCON station.
The \ch{XCO2} series has a 
clear seasonal cycle with a peak around February, before the photosynthesis
kicks in. The \ch{XCH4} also exhibit a
seasonal behaviour, with a peak in \ch{XCH4} around November, however the
\ch{XCH4} has more variation. The \ch{XCH4} in November is probably due to the
sink being weaker due to less sunlight, but the sources are still large, since
it is not yet cold enough for the for the organic material in the peat-lands to
be frozen. 

The average annual growth rate during this period for \ch{XCO2} and \ch{XCH4} is
about 2.2 ppm/year and 5778 ppb/year respectively.  
 
\begin{figure}[htbp]
    \includegraphics[width=\textwidth]{../Time_series.pdf}
    \centering
    \caption{Time series of \ch{XCO2} \textbf{(a)}, and \ch{XCH4} \textbf{(b)} from 2013 until 2020, from the Caltec TCCON station.}
    \label{fig:Time_series_XCH4_XCO2}

\end{figure}

\subsection*{b)}
\Cref{fig:scatter_XCO2} shows a scatter plot of \ch{XCO2} from the GOSAT
satellite and compared against the ground based observations from the Caltec
TCCON station using two different geometric co-locations. The the small
co-location corresponds to 10  
\begin{figure}[htbp]
    \centering
    \includegraphics[width=\textwidth]{../scatter_plot_XCO2.pdf}
    \caption{Scatter plot of daily mean \ch{XCO2} from GOSAT (x-axis) and TCCON caltec (y-axis). The red dotted is a least squares regression line of the data. \textbf{(a)}, corresponds to a tight geometric co-location, whereas \textbf{(b)} corresponds to a wider geometric co-location}
    \label{fig:scatter_XCO2}
\end{figure}

\textbf{Codes Here}\footnote{https://github.com/Ovewh/ATM325/tree/master/Exercise5}