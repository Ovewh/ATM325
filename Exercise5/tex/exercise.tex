\section*{1.}

\subsubsection*{a)} \Cref{fig:Time_series_XCH4_XCO2} shows a time series of
daily average total column \ch{CO2} and \ch{CH4} from the Caltec TCCON station.
The \ch{XCO2} series has a 
clear seasonal cycle with a peak around February, before the photosynthesis
kicks in. The \ch{XCH4} also exhibit a
seasonal behaviour, with a peak in \ch{XCH4} around November, however the
\ch{XCH4} series has more variation. 
%The \ch{XCH4} peak in November is probably due to the
%sink being weaker due to less sunlight, but the sources are still large, since
%it is not yet cold enough for the for the organic material in the peat-lands to
%be frozen. 

The average annual growth rate during this period for \ch{XCO2} and \ch{XCH4} is
about 2.2 ppm/year and 5778 ppb/year respectively.  
 
\begin{figure}[htbp]
    \includegraphics[width=\textwidth]{../Time_series.pdf}
    \centering
    \caption{Time series of \ch{XCO2} \textbf{(a)}, and \ch{XCH4} \textbf{(b)} from 2013 until 2020, from the Caltec TCCON station.}
    \label{fig:Time_series_XCH4_XCO2}

\end{figure}

\subsection*{b)}
\Cref{fig:scatter_XCO2} shows a scatter plot of \ch{XCO2}, where observations
from the GOSAT satellite is compared to ground based observations from
the Caltec Total Carbon Column Observing Network (TCCON) station using two
different geometric co-locations. The instrument used at Caltec TCCON is a
Fourier-transform spectroscopy which measures radiances from direct
solar radiation. The tight co-location is $\pm$ 5 \degree in
longitude and $\pm$ 2.5 \degree in latitude from the TCCON site and the wider
co-location is $\pm$ 10 \degree in longitude and $\pm$
5 \degree in latitude.

Generally the GOSAT retrievals
are underestimating the \ch{XCO2} with a mean bias of -0.873 ppm and -0.955 ppm for the
tight and wide co-location respectively. The marginal difference in bias between
the small and large co-location relates to how well mixed \ch{CO2} is in the
atmosphere. The performance of the linear
regression is marginally worse for the tight co-location due to there being less
points. The correlation between GOSAT and TCCON is quite high for both the tight
and wide co-location with correlation coefficients of 0.939, and 0.942
respectively. Looking at the
standard deviation the for the GOSAT observations within the two  co-location, they
are more or less equal, being 4.776 ppm and 4.839 ppm for the tight and wide
co-locations respectively. 
\begin{figure}[htbp]
    \centering
    \includegraphics[width=\textwidth]{../scatter_plot_XCO2.pdf}
    \caption{Scatter plot of daily mean \ch{XCO2} from GOSAT (x-axis) and TCCON caltec (y-axis). The red dotted is a least squares regression line of the data. \textbf{(a)}, corresponds to a tight geometric co-location, whereas \textbf{(b)} corresponds to a wider geometric co-location}
    \label{fig:scatter_XCO2}
\end{figure}

\subsection*{c)} 
\Cref{fig:scatter_XCH4} shows a scatter plot of \ch{XCH4}, where observations
from the GOSAT satellite is compared against ground based observations from
the Caltec TCCON station using two different geometric co-locations, as in the
previous section. In contrast to the \ch{XCO2} validation, the \ch{XCH4}
validation shows a major difference between the tight co-location and wide
co-location. The tight 
co-location has a much smaller bias of -2.464 ppb compared to the -8.741 ppb for
the wide co-location. The large difference in bias between the two co-locations
suggest that there are local sources and sinks of \ch{CH4} close to the TCCCON
station which is not captured by the wider co-location. The standard deviation
is also higher for the \ch{XCH4} retrials, hinting to higher variability in
\ch{CH4} compared to \ch{CO2} concentrations. 
\begin{figure}
    \centering
    \includegraphics[width=\textwidth]{../scatter_plot_XCH4.pdf}
    \caption{Scatter plot of daily mean \ch{XCH4} from GOSAT (x-axis) and TCCON caltec (y-axis). The red dotted is a least squares regression line of the data. \textbf{(a)}, corresponds to a tight geometric co-location, whereas \textbf{(b)} corresponds to a wider geometric co-location}
    \label{fig:scatter_XCH4}
\end{figure}

\section*{2.}
The animation of daily global \ch{CO} Mole fraction retrived from the AIRS
instrument on board the Aqua satellite during August and September 2019
is available on
\href{https://github.com/Ovewh/ATM325/tree/master/Exercise5/CO_Global_2019_Aug.gif}{GitHub}
There are areas with enhanced \ch{CO} Mole air fraction 
over Russia in Siberia, during the first three weeks of August. Which is the due
to large forest fire in the area at the time. There is also enhanced CO over
central Africa during the whole two months period which is also caused by a
large forest fire. The last region with enhanced \ch{CO} is over the Amazon rain
forest, the burning of the Amazon rain forest got a lot of media attention last
year.  
 

\textbf{Codes Here}\footnote{https://github.com/Ovewh/ATM325/tree/master/Exercise5}
