\section*{1.}

\subsubsection*{c)}

The most noticeable difference between the quality check applied
(\Cref{fig:qa_xco2}) and no quality check (\Cref{fig:no_qa_xco2})
is the smaller standard deviation within each latitude bin. Around half of the
data points is discarded by the quality check. The latitudinal variation in
\ch{XCO2} is more apparent in when the quality check is applied, removing the
unrealistic values of \ch{XCO2}. The reason why the quality discarded so many
data points, I would guess is mostly due to clouds interfering with the signal,
but at the high latitudes points might also been removed due to ice and snow
cover on the ground. By looking at MODIS Aqua true color images from NASA
Worldview, there seem to be a lot of clouds between 30 - 40\degree N, which
could be the reason why the around half of the data points in that bin was
discarded.   
\begin{figure}[htbp]
    \includegraphics[width=\textwidth]{../xCO2.pdf}
    \centering
    \caption{Latitudinal mean  bias corrected \ch{XCO2} from the OCO2 satellite from 28.04.2016 (left) 
    and 28.04.2017 (right), without quality flags applied. The upper panels 
    show the number of data points in each bin. The lower panels shows the \ch{XCO2} 
    concentration in ppm on the y-axis, along the x-axis is the upper limit of
    the latitude bins. The error bars represent $\pm$ one standard deviation.}
    \label{fig:no_qa_xco2}

\end{figure}

\begin{figure}[htbp]
    \centering
    \includegraphics[width=\textwidth]{../qa_xCO2.pdf}
    \caption{Latitudinal mean bias corrected XCO2 from the OCO2 satellite from 28.04.2016 (left) 
    and 28.04.2017 (right), with quality flags applied. The upper panels 
    show the number of data points in each bin. The lower panels shows the \ch{XCO2} 
    concentration in ppm on the y-axis, along the x-axis is the upper limit of
    the latitude bins. The error bars represent $\pm$ one standard deviation.}
    \label{fig:qa_xco2}

\end{figure}


\section*{2.}

\subsection*{a)}

Looking at the latitudinal variation in \ch{XCO2} (\Cref{fig:qa_xco2}) there is about a 7 ppm
difference between the northern and southern hemisphere. This difference between
the hemispheres is due to most of the sources of \ch{CO2} is located in the
northern hemisphere. The data is also from spring time, so the photosynthesis
have not been able to even out the difference in \ch{CO2} concentration between
the hemispheres yet. 

\subsection*{b)}

\begin{figure}[htbp]
    \centering
    \includegraphics[width=0.7\textwidth]{../deltaXco2.pdf}
    \caption{Difference in mean Bias corrected XCO2 from the OCO2 satellite
    between 28.04.2017 and 28.04.2016, with quality flags applied. The
    difference in \ch{XCO2} between the years in ppm on the
    y-axis, along the x-axis is the upper limit of
    the latitude bins.}
    \label{fig:Deltaqa_xco2}

\end{figure}


Looking at inter-annual difference, there is a over all increase in mean
\ch{CO2} concentration in 2017 of 2.218 ppm compared to the previous year. The
largest increase is in the high latitude in the northern hemisphere where it is
about 3 ppm. The difference is at each latitude band is caused by our
increasing consumption of fossil fuels. 

\textbf{Codes Here}\footnote{https://github.com/Ovewh/ATM325/tree/master/Exercise4}